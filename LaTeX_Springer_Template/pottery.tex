%%%%%%%%%%%%%%%%%%%%%%% file template.tex %%%%%%%%%%%%%%%%%%%%%%%%%
%
% This is a general template file for the LaTeX package SVJour3
% for Springer journals.          Springer Heidelberg 2010/09/16
%
% Copy it to a new file with a new name and use it as the basis
% for your article. Delete % signs as needed.
%
% This template includes a few options for different layouts and
% content for various journals. Please consult a previous issue of
% your journal as needed.
%
%%%%%%%%%%%%%%%%%%%%%%%%%%%%%%%%%%%%%%%%%%%%%%%%%%%%%%%%%%%%%%%%%%%
%
% First comes an example EPS file -- just ignore it and
% proceed on the \documentclass line
% your LaTeX will extract the file if required
\begin{filecontents*}{example.eps}
%!PS-Adobe-3.0 EPSF-3.0
%%BoundingBox: 19 19 221 221
%%CreationDate: Mon Sep 29 1997
%%Creator: programmed by hand (JK)
%%EndComments
gsave
newpath
  20 20 moveto
  20 220 lineto
  220 220 lineto
  220 20 lineto
closepath
2 setlinewidth
gsave
  .4 setgray fill
grestore
stroke
grestore
\end{filecontents*}
%
\RequirePackage{fix-cm}
%
%\documentclass{svjour3}                     % onecolumn (standard format)
%\documentclass[smallcondensed]{svjour3}     % onecolumn (ditto)
\documentclass[smallextended]{svjour3}       % onecolumn (second format)
%\documentclass[twocolumn]{svjour3}          % twocolumn
%
\smartqed  % flush right qed marks, e.g. at end of proof
%
\usepackage{graphicx}
%
% \usepackage{mathptmx}      % use Times fonts if available on your TeX system
%
% insert here the call for the packages your document requires
%\usepackage{latexsym}
% etc.
%
% please place your own definitions here and don't use \def but
% \newcommand{}{}
%
% Insert the name of "your journal" with
% \journalname{myjournal}
%
\begin{document}

\title{PotteryVR: a Realtime Virtual Pottery Tool%\thanks{Grants or other notes
%about the article that should go on the front page should be
%placed here. General acknowledgments should be placed at the end of the article.}
}
%\subtitle{Do you have a subtitle?\cite{Jacob2008Reality}  \\ If so, write it here \cite{website:unity3d}}

%\titlerunning{Short form of title}        % if too long for running head

\author{Zihan Gao         \and
        Guangsheng Feng %etc.
}

%\authorrunning{Short form of author list} % if too long for running head

\institute{F. Author \at
              first address \\
              Tel.: +123-45-678910\\
              Fax: +123-45-678910\\
              \email{fauthor@example.com}           %  \\
%             \emph{Present address:} of F. Author  %  if needed
           \and
           S. Author \at
              second address
}

\date{Received: date / Accepted: date}
% The correct dates will be entered by the editor


\maketitle

\begin{abstract}
We present DigiClay, an interactive virtual reality (VR) modeling system that allows novice users to create virtual pottery works with their bimanual movement using hand-held motion controllers.
Our system consists of two major components: a mesh generator and an interactive pottery model editor.
The mesh generator can procedurally generate a realistic clay mesh. With the interactive pottery model editor, the user can shape the clay mesh in realtime intuitively to create a virtual pottery work.
The virtual pottery created by our system can be exported as an obj file and used for 3D printing.
The results of our user study have shown that our system is easier to use compared with traditional modeling systems. Users without real life pottery making experience and 3D modeling knowledge can easily create pottery works with our system.\cite{website:letspottery}


Insert your abstract here. Include keywords, PACS and mathematical
subject classification numbers as needed.
\keywords{First keyword \and Second keyword \and More}
% \PACS{PACS code1 \and PACS code2 \and more}
% \subclass{MSC code1 \and MSC code2 \and more}
\end{abstract}

%%% Introduction

\section{Introduction}
\label{sec:1}
%[CAD tools - hard to use]
Pottery is one of the oldest inventions in heterogeneous civilizations in human history for thousands of years, in which the traditional creation process is called "throwing", where a ball of clay is placed on a pottery wheel and shaped by hands. In recent years, emerging technologies such as 3D printing introduces a new way of pottery design, enabling people to fabricate pots in a digital manner with the help of Computer Aided Design (CAD) software and 3D printers. There are several desktop CAD systems either from academic or industry, some of them are developed specifically for pottery design[a, b], which can generate 3D meshes based on the values from user keyboard and mouse input. However, for novice users who are not professional 3D artists, these systems are formidable to learn due to complex user interfaces. This kind of systems limits the creativity of novice users, since it is quite challenging for them to master the tool in a short period of time. In addition, the experience of creating potteries using CAD tools is far different from in reality.

%[in-air interaction system - freehand, lacks robustness, feedback and reality]
Traditional CAD tools, such as Maya[] and 3ds Max[], are formidable to learn due to complex user interfaces. To address this situation, some camera-based virtual pottery systems such as [a, b, c] have been developed, which provides much simpler user interfaces, allowing users to design pots using freehand. These works indeed provide a gentle learning curve, however, they have some common limitations. First, freehand interactions from depth cameras lacks robustness due to jittering, whose inaccuracy hinders user experience severely. Second, freehand interaction lacks haptic feedback, making it difficult for users to perceive if they have touched anything in VR environments. In addition, some realistic features of clay, namely shape irregularity, thickness and etc., are missing from these systems, undermining the look and feel in the pot design process.

%[design goals: 1. simple 2. robust/realistic]
In this paper, we present PotteryVR, a VR system that allows users to design virtual pottery models from their hand movement using hand-held motion controllers. There are two major design goals for PotteryVR:
%[easy]
1 Provide a simple and intuitive interface for novice users to learn virtual pottery skills with minimum cognitive load.
%[robust realistic]
2 Design a robust virtual pottery system on VR devices that can generate realistic clay meshes and provide refined haptic feedbacks based on bimanual spatial interactions.

%[design considerations: 1.HCD 2. RBI]
Oviatt \cite{oviatt2006human} concluded that human-centered design can minimize users’ cognitive load, which effectively frees up mental resources for performing better while also remaining more attuned to the world around them. So we take the human-centered design approach, which models users’ natural behavior to begin with so that interfaces can be designed that are more intuitive, easier to learn, and freer of performance errors. In addition, Jacob et al. \cite{Jacob2008Reality} summarized that the designer's goal should be to allow the user to perform realistic tasks realistically, to provide additional non real-world functionality, and to use analogies for these commands whenever possible. Hence,while we design our system based on pottery creation process in reality to minimize the effort, we should provide convenient functionality in our system for efficiency.

%[contributions]
The main contributions of our works are:

1 Present a virtual pottery system which can generate pot models from user spatial interaction for 3D printing.

2 Propose a virtual pottery workflow by introducing simple and intuitive user interfaces for modeling, enabling novice users to understand and learn pottery production pipeline.

3 Conduct an user study showing the comparison results among three interaction systems. The results have shown that our system is easier to use compared with traditional 3D modeling tools and is more intuitive and immersive than touchscreen based interfaces.


%%%


\section{Related Work}
\label{sec:2}

\subsection{Bimanual Interaction}
\label{sec:2.1}
Bimanual interaction has been a popular research field, which can accomplish a variety of tasks in both physical and virtual environments. In terms of mechanisms, bimanual interaction can be classified into two categories: bare-hand based interactions and instrument based interactions. There are a great number of research efforts \cite{walter2014cuenesics,cui2016exploration,ramani2015gesture,murugappan2013handy,han2014virtual} on bare-hand based interactions using depth camera such as Kinect, Leap Motion etc. Cuenestics \cite{walter2014cuenesics} is a design space for hand-gesture based mid-air selection techniques using a depth camera Kinect, where users can select contents on interactive public displays with their gesture input. Cui et al. \cite{cui2016exploration} proposed a modeling system with natural free-hand interaction using a Leap Motion controller, allowing users to grab and manipulate objects with one or two hands intuitively. While these works provided accessibility to users, the limitations are obvious as well: The input is not always accurate due to many factors such as lighting condition and occlusion, which might cause user frustration. In addition, these methods do not provide haptic feedbacks, which hinders the realistic feel for users. Unlike bare-hand based interactions, instrument based interactions provide more control precision, haptic feedback and unambiguity. Surface Drawing \cite{schkolne2001surface} is a system for creating organic 3D shapes using tangible tools such as gloves, where users can define strokes with the path of hands wearing gloves. Hinckley et al. \cite{hinckley1998two} did a research on two-handed virtual manipulation with a point design of a prop-based system, which allows users to view a cross section of a brain with interface props. These works have a common problem that the usage of these instruments are limited to a lab context that very few users can access. There are a number of motion controllers such as Wii Remote \cite{wingcrave2010wii} and HTC Vive \cite{niehorster2017accuracy} that become widely commercialized products and accessible to consumers, as well as used in scientific research. In our project, HTC Vive system is used in our system, which provides precision, haptic feedback and well accessed by consumers.

\subsection{Artistic Tools in VR}
\label{sec:2.2}
Virtual Reality has shown great potential for art and design, which not only provides immersive and intuitive interfaces for user, but also creates new art medium, new art form and novel experience\cite{laviola20113d}. CavePainting \cite{keefe2001cavepainting} is a 3D artistic medium in a fully immersive environment, which enables artists to create spatial paintings with physical props and gestures. Agrawala et al.\cite{agrawala19953d} developed an interface for painting on polygon meshes using a 6DOF space tracker, which provides a natural force-feedback for painting, allowing users to place colors on meshes intuitively. MAI Painting Brush++ \cite{otsuki2017brush} is a brush device for virtual painting of 3D virtual objects, where users could take a physical object in the real world and apply virtual paint to it with visual and haptic feedback. Virtual Clay \cite{mcdonnell2001virtual} is a sculpture framework based on subdivision solids and physics-based modeling, which is equipped with natural, haptic based interaction, providing users with a realistic sculpting experience. Sheng et al. \cite{sheng2006interface} proposed an interface for virtual 3D sculpting, which uses camera-based motion tracking technology to track passive markers on the fingers and prop, enabling users to apply operations such as deforming, smoothing, pasting and extruding.

\subsection{Virtual Pottery Systems}
\label{sec:2.3}


Several systems have been developed for virtual pottery, unlike other sculpting systems mentioned above, v

Qp \cite{koutsoudis2009qp} was a tool for generating 3D pottery models, which can produce a collection of random 3D ancient greek vessels.

Based on number-theoretic techniques, Kumar et al. \cite{kumar2011wheel} presented a system for creating digital potteries including thick-walled potteries as well, which resembles pottery works in real life.

While these systems can generate heterogeneous pottery models efficiently, their user interfaces are limited to traditional keyboard and mouse input, which are not helpful for users to understand the pottery creation process.

%%CHINA korida1997interactive

Handy-Potter \cite{murugappan2013handy} was a rapid 3D creation tool, which tracks user skeletons with depth sensing camera Kinect, enabling users to create potteries using hands and arms.

%%CAD ramani2015gesture

Han et al. \cite{han2014virtual} presented an audiovisual interface, where hand motions are translated into musical sound.

In AR Pottery \cite{han2007ar}, augmented reality has been applied to pottery design, with which users can deform a virtual pottery using a marker held by hand.

Although with these systems users could create some virtual pottery works, the actions applied are quite different from real life pottery making process. Thus, users cannot learn the actual pottery process from using these systems.

[design consideration - EDUCATION!!!]

In contrast to existing works, our system provides a novel pottery creation workflow in virtual reality which lets user shape and color pottery through two-handed spatial interactions, helping novice users to understand and learn real life pottery.



\section{Workflow}
\label{sec:1}

%Text with citations \cite{RefB} and \cite{RefJ}.
\subsection{Subsection title}
\label{sec:2}
as required. Don't forget to give each section
and subsection a unique label (see Sect.~\ref{sec:1}).
\paragraph{Paragraph headings} Use paragraph headings as needed.
\begin{equation}
a^2+b^2=c^2
\end{equation}









To illustrate the pipeline of pottery creation in our system, an example workflow using DigiClay is described as follows.

When a user starts to use DigiClay, a realistic clay mesh is generated with Perlin noise. In order to control the height of the clay, the user can use both of her hands to move up or down together. With one hand or both hands, the user can deform the clay symmetrically with some tunable parameters. The user can also smooth the clay during the process to get rid of sharp features.

The clay can be baked into a pottery when the user chooses to do so. Then the user can apply color painting on the pottery model. While the non-dominant hand of the user can manipulate the model such as translation and rotation, the dominant hand can pick colors and apply painting on the pottery. After the creation is done, the user can save the screenshot and save the pottery model as an OBJ file, which can be used for 3D printing.












\section{Implementation}
\label{sec:1}

In this section, we will describe more details about the implementation. 



\subsection{Mesh Generation}
\label{sec:2}

In the initialization phase, our system procedurally generates a mesh for the clay. In our approach, we approximate the initial shape of the clay on the pottery wheel as a frustum, then adding Perlin noises to make the clay realistic. The basic parameters of the frustum is described in Table [1].


\subsection{Mesh Editing}
\label{sec:2}
After observing and analyzing several real-life pottery-making videos, we put mesh editing operation into four categories: height control, thickness control, mesh deformation and mesh smoothing. These operations will be discussed in detail in the following sections.

\subsubsection{Height Control}
\label{sec:3}

\subsubsection{Thickness Control}
\label{sec:3}

\subsubsection{Mesh Shaping}
\label{sec:3}

\subsubsection{Mesh Smoothing}
\label{sec:3}

\paragraph{Radial Smoothing}

\paragraph{Laplacian Smoothing}


\subsection{Haptic Feedback}
\label{sec:2}







\section{Results}
\label{sec:1}

Unity3D and HTC Vive
Windows 10

Mesh Generation Parameters




\section{User Study}
\label{sec:1}

\subsection{Evaluated Systems}
\label{sec:2}

(description of each system)
DigiClay – a pottery tool in VR

Let’s Create Pottery – a pottery tool on mobile devices

Maya – traditional modeling tool

Reasons – why choosing these three?



\subsection{Participants}
\label{sec:2}

15-20

Male Female

Age range

Familiarity with VR systems

Traditional CAD tools 

Real pottery experience



\subsection{Experimental Design and Procedure}
\label{sec:2}

10 min training

3 tasks

Task\textsubscript{DigiClay}

Task\textsubscript{LetsPottery}

Task\textsubscript{Maya}

Task details

NASA-TLX

Questions

Q1
…
Q5

Some explanations




\subsection{Study Results}
\label{sec:2}

Figure - NASA-TLX

Figure - Questions

Findings from figures


\subsection{User Feedbacks}
\label{sec:2}

Positive

Suggestions



\section{Discussions}
\label{sec:1}

Limitations

External 

Internal

Other…




\section{Conclusions}
\label{sec:1}




% For one-column wide figures use
\begin{figure}
% Use the relevant command to insert your figure file.
% For example, with the graphicx package use
  \includegraphics{example.eps}
% figure caption is below the figure
\caption{Please write your figure caption here}
\label{fig:1}       % Give a unique label
\end{figure}
%
% For two-column wide figures use
\begin{figure*}
% Use the relevant command to insert your figure file.
% For example, with the graphicx package use
  \includegraphics[width=0.75\textwidth]{example.eps}
% figure caption is below the figure
\caption{Please write your figure caption here}
\label{fig:2}       % Give a unique label
\end{figure*}
%
% For tables use
\begin{table}
% table caption is above the table
\caption{Please write your table caption here}
\label{tab:1}       % Give a unique label
% For LaTeX tables use
\begin{tabular}{lll}
\hline\noalign{\smallskip}
first & second & third  \\
\noalign{\smallskip}\hline\noalign{\smallskip}
number & number & number \\
number & number & number \\
\noalign{\smallskip}\hline
\end{tabular}
\end{table}


\begin{acknowledgements}
%If you'd like to thank anyone, place your comments here
%and remove the percent signs.
\end{acknowledgements}

% BibTeX users please use one of
%\bibliographystyle{spbasic}      % basic style, author-year citations
\bibliographystyle{spmpsci}      % mathematics and physical sciences
%\bibliographystyle{spphys}       % APS-like style for physics
\bibliography{pottery}   % name your BibTeX data base

% Non-BibTeX users please use
%\begin{thebibliography}{}
%
% and use \bibitem to create references. Consult the Instructions
% for authors for reference list style.
%
%\bibitem{RefJ}
% Format for Journal Reference
%Author, Article title, Journal, Volume, page numbers (year)
% Format for books
%\bibitem{RefB}
%Author, Book title, page numbers. Publisher, place (year)
% etc
%\end{thebibliography}

\end{document}
% end of file template.tex

